% Created 2011-08-27 Sat 18:06
\documentclass[11pt]{article}
\usepackage[utf8]{inputenc}
\usepackage[T1]{fontenc}
\usepackage{graphicx}
\usepackage{longtable}
\usepackage{hyperref}


\title{The Ally programming language}
\author{Derek A. Rhodes}
\date{27 August 2011}

\begin{document}

\maketitle

\setcounter{tocdepth}{3}
\tableofcontents
\vspace*{1cm}

\usepackage[bigsym]{mathptm}
\sffamily 
\DeclareOption{bigsym}{\DeclareSymbolFont{largesymbols}{OMX}{psycm}{m}{n}}
\ProcessOptions

\section{What is it?}
\label{sec-1}

The ally grammar describes specialized petrinets that execute general
programs which

 are isomorphic to diagrams.

\subsection{What are.}
\label{sec-1.1}

Petrinets effectively describe concurrent systems.

There are n elements

There are tokens, functions and places.
a function pulls one or more tokens from one of more places, uses it
as an argument and puts the result in one or more places
 
\subsection{Syntax.}
\label{sec-1.2}


$\times$
\rhd

\begin{eq}

a \triangleright b \triangleright c \triangleright d \triangleright e
\end{eq}

predicate arrows.  These inspect the places before them and put a
token in one of two control places.  

+>
->
<>


<+
+>
->
<-
<>
<<
>>

\begin{center}
\begin{tabular}{l}
 <+  \\
 +>  \\
\hline
 <-  \\
 <>  \\
 <<  \\
 >>  \\
 >   \\
 <   \\
\end{tabular}
\end{center}




\end{document}